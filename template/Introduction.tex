\documentclass[root.tex]{subfiles}

\begin{document}

{\pagestyle{empty}}
\section{Introduction}
\label{chap:introduction}
The driving behaviour of \gls{HCT} vehicles is in many ways different to that of single unit trucks and needs to be researched in great detail to gain an understanding of the vehicle's dynamic properties, that is equally detailed as it is for other vehicle classes. This will lead to development of better safety and assistance systems and thus reduce threat potential, accidents and fatalities involving this emerging mode of transportation. 

The research project in which this work is embedded aims to develop an active dolly, meaning that steering will be autonomously conducted by the dolly based on the driving situation at hand and various vehicle parameters (e.g. speed, steering wheel angle).
This abstract control algorithm will be executed on a rapid-prototyping system which is linked to and controls the dolly. To supply this connection between the hardware and control-algorithm implemented in the modeling-environment Simulink is the main-contribution of this work. 


The following points will be covered in this paper:
\begin{itemize}
	\item hardware and software utilized to achieve \gls{HIL} verification for an active converter dolly
	\item evaluation of existing delays in the implementation and their consequences
	\item discussion of three standard maneuvers for \gls{HCT}-combinations executed on the  developed \gls{HIL}-system using the aforementioned controllers and a comparison with simulation results
	\item propose necessary changes to the set-up for taking the developed solution to the test-track in the future
\end{itemize}

\end{document}