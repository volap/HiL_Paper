%%%%%%%%%%%%%%%%%%%%%%%%%%%%%%%%%%%%%%%%%%%%%%%%%%%%%%%%%%%%%%%%%%%%%%%%%%%%%%%%
%2345678901234567890123456789012345678901234567890123456789012345678901234567890
%        1         2         3         4         5         6         7         8

%\documentclass[letterpaper, 10 pt, conference]{ieeeconf}  % Comment this line out if you need a4paper

\documentclass[letterpaper, 10pt, conference]{IEEEconf}      % Use this line for a4 paper

\IEEEoverridecommandlockouts                              % This command is only needed if 
                                                          % you want to use the \thanks command

\overrideIEEEmargins                                      % Needed to meet printer requirements.

% See the \addtolength command later in the file to balance the column lengths
% on the last page of the document

% The following packages can be found on http:\\www.ctan.org
\usepackage{graphics} % for pdf, bitmapped graphics files
\usepackage{epsfig} % for postscript graphics files
\usepackage{mathptmx} % assumes new font selection scheme installed
\usepackage{times} % assumes new font selection scheme installed
\usepackage{amsmath} % assumes amsmath package installed
\usepackage{amssymb}  % assumes amsmath package installed
\usepackage[acronym]{glossaries}
\usepackage{subfiles}
% new added
%\usepackage{fixltx2e}
%\usepackage{dblfloatfix}
%\usepackage{geometry}% http://ctan.org/pkg/geometry
%\usepackage{lipsum}% http://ctan.org/pkg/lipsum
\usepackage{siunitx}
\usepackage{graphicx}
\usepackage{epstopdf}
\usepackage{tikz}
\usepackage{hyperref}
\usepackage{bookmark}
\makeglossaries

%\renewcommand\@seccntformat[1]{}

\usetikzlibrary{arrows,positioning,shapes.geometric}

\DeclareGraphicsExtensions{.eps}
\subfile{glossary}

\title{\LARGE \bf
Development and evaluation of an experimental platform for steered axles of long combination vehicles}

\author{Michael Hofmann$^{1}$ and Sebastian Franz$^{1}$% <-this % stops a space
	%\thanks{*This work was supported by Volvo Group Trucks Technology, Chalmers University of Technology, SAFER, Parator.}% <-this % stops a space
	\thanks{$^{1}$Vehicle Engineering and Autonomous Systems, Department of Applied Mechanics, Chalmers University of Technology, SE-412~96 G\"OTEBORG, Sweden,
		{\tt\small michael.hofmann@alumni.chalmers.se},
		{\tt\small sebastian.franz@alumni.chalmers.se}}%
}





\begin{document}



\maketitle
\thispagestyle{empty}
\pagestyle{empty}


%%%%%%%%%%%%%%%%%%%%%%%%%%%%%%%%%%%%%%%%%%%%%%%%%%%%%%%%%%%%%%%%%%%%%%%%%%%%%%%%
\begin{abstract}

\gls{HCT}-vehicles require different strategies in controlling lateral dynamics of the combinations to ensure optimal paths taken by the trailers. Promising steering algorithms actuating more then one axles of the combination have been developed in previous works !!!!MI-paper quote!!!!! and now need to be verified in the field. This work thus developed an experimentation platform incorporating a rapid-prototyping system to provide the possibility of evaluating these algorithms. In this publication the solution is detailed as a \gls{HIL}-platform linking a vehicle dynamics frame-work with two steered axles. In accordance with the automotive development process after the V-model, this allows to safely verify the functioning of both software and hardware before performing track-tests of the fully integrated system. This paper outlines the development and capabilities of the resulting experimental platform and gives a short example of its performance in a standard-maneuver, which is also used to proof the validity between simulation and \gls{HIL}-environment enabling full system testing.

\end{abstract}


%%%%%%%%%%%%%%%%%%%%%%%%%%%%%%%%%%%%%%%%%%%%%%%%%%%%%%%%%%%%%%%%%%%%%%%%%%%%%%%%
\subfile{Introduction}
\subfile{Hardware-setup}
\subfile{Software-setup}
\subfile{HiL-architecture}
\subfile{Delays}
\subfile{Showcase_Maneuver}
\subfile{Results}

\subfile{Conclusion}
\subfile{Related_works}
\subfile{Appendix}
\subfile{Acknowledgement}




\addtolength{\textheight}{-12cm}   % This command serves to balance the column lengths
                                  % on the last page of the document manually. It shortens
                                  % the textheight of the last page by a suitable amount.
                                  % This command does not take effect until the next page
                                  % so it should come on the page before the last. Make
                                  % sure that you do not shorten the textheight too much.

%%%%%%%%%%%%%%%%%%%%%%%%%%%%%%%%%%%%%%%%%%%%%%%%%%%%%%%%%%%%%%%%%%%%%%%%%%%%%%%%



%%%%%%%%%%%%%%%%%%%%%%%%%%%%%%%%%%%%%%%%%%%%%%%%%%%%%%%%%%%%%%%%%%%%%%%%%%%%%%%%



%%%%%%%%%%%%%%%%%%%%%%%%%%%%%%%%%%%%%%%%%%%%%%%%%%%%%%%%%%%%%%%%%%%%%%%%%%%%%%%%


%%%%%%%%%%%%%%%%%%%%%%%%%%%%%%%%%%%%%%%%%%%%%%%%%%%%%%%%%%%%%%%%%%%%%%%%%%%%%%%%

%References are important to the reader; therefore, each citation must be complete and correct. If at all possible, references should be commonly available publications.



\begin{thebibliography}{99}

\bibitem{c27} M.S. Kati, J. Fredriksson, L. Laine, B. Jacobson,``Performance Improvement for A-double Combination by introducing a Smart Dolly," in Proceedings of the 13th International Heavy Vehicle Transport Technology Symposium, San Luis, Argentina, 2014.

\bibitem{nilsson2015traffic},
	title={On Traffic Situation Predictions for Automated Driving of Long Vehicle Combinations},
	author={Nilsson, Peter},
	year={2015},
	publisher={Chalmers University of Technology}


%\bibitem{c1} G. O. Young, �Synthetic structure of industrial plastics (Book style with paper title and editor),� 	in Plastics, 2nd ed. vol. 3, J. Peters, Ed.  New York: McGraw-Hill, 1964, pp. 15�64.
%\bibitem{c2} W.-K. Chen, Linear Networks and Systems (Book style).	Belmont, CA: Wadsworth, 1993, pp. 123�135.
%\bibitem{c3} H. Poor, An Introduction to Signal Detection and Estimation.   New York: Springer-Verlag, 1985, ch. 4.
%\bibitem{c4} B. Smith, �An approach to graphs of linear forms (Unpublished work style),� unpublished.
%\bibitem{c5} E. H. Miller, �A note on reflector arrays (Periodical style�Accepted for publication),� IEEE Trans. Antennas Propagat., to be publised.
%\bibitem{c6} J. Wang, �Fundamentals of erbium-doped fiber amplifiers arrays (Periodical style�Submitted for publication),� IEEE J. Quantum Electron., submitted for publication.
%\bibitem{c7} C. J. Kaufman, Rocky Mountain Research Lab., Boulder, CO, private communication, May 1995.
%\bibitem{c8} Y. Yorozu, M. Hirano, K. Oka, and Y. Tagawa, �Electron spectroscopy studies on magneto-optical media and plastic substrate interfaces(Translation Journals style),� IEEE Transl. J. Magn.Jpn., vol. 2, Aug. 1987, pp. 740�741 [Dig. 9th Annu. Conf. Magnetics Japan, 1982, p. 301].
%\bibitem{c9} M. Young, The Techincal Writers Handbook.  Mill Valley, CA: University Science, 1989.
%\bibitem{c10} J. U. Duncombe, �Infrared navigation�Part I: An assessment of feasibility (Periodical style),� IEEE Trans. Electron Devices, vol. ED-11, pp. 34�39, Jan. 1959.
%\bibitem{c11} S. Chen, B. Mulgrew, and P. M. Grant, �A clustering technique for digital communications channel equalization using radial basis function networks,� IEEE Trans. Neural Networks, vol. 4, pp. 570�578, July 1993.
%\bibitem{c12} R. W. Lucky, �Automatic equalization for digital communication,� Bell Syst. Tech. J., vol. 44, no. 4, pp. 547�588, Apr. 1965.
%\bibitem{c13} S. P. Bingulac, �On the compatibility of adaptive controllers (Published Conference Proceedings style),� in Proc. 4th Annu. Allerton Conf. Circuits and Systems Theory, New York, 1994, pp. 8�16.
%\bibitem{c14} G. R. Faulhaber, �Design of service systems with priority reservation,� in Conf. Rec. 1995 IEEE Int. Conf. Communications, pp. 3�8.
%\bibitem{c15} W. D. Doyle, �Magnetization reversal in films with biaxial anisotropy,� in 1987 Proc. INTERMAG Conf., pp. 2.2-1�2.2-6.
%\bibitem{c16} G. W. Juette and L. E. Zeffanella, �Radio noise currents n short sections on bundle conductors (Presented Conference Paper style),� presented at the IEEE Summer power Meeting, Dallas, TX, June 22�27, 1990, Paper 90 SM 690-0 PWRS.
%\bibitem{c17} J. G. Kreifeldt, �An analysis of surface-detected EMG as an amplitude-modulated noise,� presented at the 1989 Int. Conf. Medicine and Biological Engineering, Chicago, IL.
%\bibitem{c18} J. Williams, �Narrow-band analyzer (Thesis or Dissertation style),� Ph.D. dissertation, Dept. Elect. Eng., Harvard Univ., Cambridge, MA, 1993. 
%\bibitem{c19} N. Kawasaki, �Parametric study of thermal and chemical nonequilibrium nozzle flow,� M.S. thesis, Dept. Electron. Eng., Osaka Univ., Osaka, Japan, 1993.
%\bibitem{c20} J. P. Wilkinson, �Nonlinear resonant circuit devices (Patent style),� U.S. Patent 3 624 12, July 16, 1990. 






\end{thebibliography}




\end{document}
