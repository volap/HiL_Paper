%%%%%%%%%%%%%%%%%%%%%%%%%%%%%%%%%%%%%%%%%%%%%%%%%%%%%%%%%%%%%%%%%%%%%%%%%%%%%%%%
%2345678901234567890123456789012345678901234567890123456789012345678901234567890
%        1         2         3         4         5         6         7         8

%\documentclass[letterpaper, 10 pt, conference]{ieeeconf}  % Comment this line out if you need a4paper

\documentclass[letterpaper, 10pt, conference]{IEEEconf}      % Use this line for a4 paper

\IEEEoverridecommandlockouts                              % This command is only needed if 
                                                          % you want to use the \thanks command

\overrideIEEEmargins                                      % Needed to meet printer requirements.

% See the \addtolength command later in the file to balance the column lengths
% on the last page of the document

% The following packages can be found on http:\\www.ctan.org
\usepackage{graphics} % for pdf, bitmapped graphics files
\usepackage{epsfig} % for postscript graphics files
\usepackage{mathptmx} % assumes new font selection scheme installed
\usepackage{times} % assumes new font selection scheme installed
\usepackage{amsmath} % assumes amsmath package installed
\usepackage{amssymb}  % assumes amsmath package installed
\usepackage[acronym]{glossaries}
\usepackage{subfiles}
% new added
%\usepackage{fixltx2e}
%\usepackage{dblfloatfix}
%\usepackage{geometry}% http://ctan.org/pkg/geometry
%\usepackage{lipsum}% http://ctan.org/pkg/lipsum
\usepackage{siunitx}
\usepackage{graphicx}
\usepackage{epstopdf}
\usepackage{tikz}
\usepackage{hyperref}
\usepackage{units}
\usepackage{bookmark}
\usepackage{csquotes}
\usepackage[firstinits=true, style=numeric, sorting=none, backend=bibtex]{biblatex}

\makeglossaries

%\renewcommand\@seccntformat[1]{}

\usetikzlibrary{arrows,positioning,shapes.geometric}

\DeclareGraphicsExtensions{.eps}
\subfile{glossary}

\title{\LARGE \bf
Development and Evaluation of an Experimental Platform for Steered Axles of Long Combination Vehicles}

\author{Michael Hofmann$^{1}$, Sebastian Franz$^{1}$, Mohammad Manjurul Islam, Leo Laine, Bengt Jacobson% <-this % stops a space
	%\thanks{*This work was supported by Volvo Group Trucks Technology, Chalmers University of Technology, SAFER, Parator.}% <-this % stops a space
	\thanks{$^{1}$Vehicle Engineering and Autonomous Systems, Department of Applied Mechanics, Chalmers University of Technology, SE-412~96 G\"OTEBORG, Sweden,
		{\tt\small michael.hofmann@alumni.chalmers.se},
		{\tt\small sebastian.franz@alumni.chalmers.se}}%
}



\bibliography{ExampleBib}

\begin{document}



\maketitle
\thispagestyle{empty}
\pagestyle{empty}


%%%%%%%%%%%%%%%%%%%%%%%%%%%%%%%%%%%%%%%%%%%%%%%%%%%%%%%%%%%%%%%%%%%%%%%%%%%%%%%%
\begin{abstract}

	\glspl{LCV} require different strategies in controlling lateral dynamics of the combinations to ensure optimal paths taken by the trailers. Promising steering algorithms actuating more than one axle of the combination have been developed in previous works and now need to be verified in real vehicle tests. This work thus developed an experimentation platform incorporating a rapid-prototyping system to provide the possibility of evaluating these algorithms on vehicle level. In this publication the solution is detailed as a \gls{HIL}-platform linking a vehicle dynamics frame-work with two steered axles. In accordance with the automotive development process after the V-model, this allows to safely verify the functioning of both software and hardware before performing track-tests of the fully integrated system with all units of a \gls{LCV}. This paper outlines the development and capabilities of the resulting experimental platform and gives a short example of its performance in a standard-maneuver, which is also used to proof the validity between simulation and \gls{HIL}-environment enabling full system testing on vehicle level.

\end{abstract}


%%%%%%%%%%%%%%%%%%%%%%%%%%%%%%%%%%%%%%%%%%%%%%%%%%%%%%%%%%%%%%%%%%%%%%%%%%%%%%%%
\subfile{Introduction}
\subfile{Hardware-setup}
\subfile{Software-setup}
\subfile{HiL-architecture}
\subfile{Delays}
\subfile{Showcase_Maneuver}
%\subfile{Results}

\subfile{Conclusion}
\subfile{Related_works}
%\subfile{Appendix}
\subfile{Acknowledgement}




\addtolength{\textheight}{-12cm}   % This command serves to balance the column lengths
                                  % on the last page of the document manually. It shortens
                                  % the textheight of the last page by a suitable amount.
                                  % This command does not take effect until the next page
                                  % so it should come on the page before the last. Make
                                  % sure that you do not shorten the textheight too much.

%%%%%%%%%%%%%%%%%%%%%%%%%%%%%%%%%%%%%%%%%%%%%%%%%%%%%%%%%%%%%%%%%%%%%%


\printbibliography

\end{document}
