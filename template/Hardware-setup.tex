\documentclass[root.tex]{subfiles}

\begin{document}

{\pagestyle{empty}}
\section{Hardware-Setup}
\label{chap:Hardware-Setup}
\subsection{Base vehicle}

The hardware base is a dolly manufactured by Parator Industries. The steering system is based around the \gls{ETS} developed and built by \gls{VSE} with two hydraulically steerable axles. It was originally meant to be used in trailer steering and as an after-market system does not tie in with any of the truck's communication networks or sensor data. This makes it OEM-independent and very robust. The \gls{ETS} solely relies on the articulation angle between the leading and following unit on which the system is mounted and the speed of the combination. The articulation angle is gathered via a dedicated sensor mounted on the king-pin, the speed-signal is gathered from the ISO-11992 \gls{CAN}.

\subsection{Rapid-prototyping system (\gls{MABII})}

To execute the readily developed algorithms\cite{c27}, that govern the steering of the \gls{HCT}-combination needed to be ported to a platform, capable of interacting with the dolly and the tractor, while ensuring robust behavior during run-time. It was decided to incorporate the \gls{MABII} by dSpace, a real-time platform for its advantages in automotive environments with a vast selection of in- and outputs for interfacing with vehicular communications systems (\gls{CAN}, ethernet, FlexLink). It conveniently ties in with Simulink, which was used for algorithm development, for code-generation. Furthermore it physically is very robust and has good logging possibilities.


\subsection{Arduino}

To convert from \gls{CAN}-messages to serial output an Arduino Due together with a MCP2551 \gls{CAN}-transceiver was used. This provided the feedback-loop from the hardware-system to the simulation environment during the \gls{HIL}-testing.

\subsection{interconnections between MABII/dolly/truck}

\end{document}