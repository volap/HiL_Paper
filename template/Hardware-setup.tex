\documentclass[root.tex]{subfiles}

\begin{document}

{\pagestyle{empty}}
\section{System Components}
\label{chap:Hardware-Setup}
\subsection{Base vehicle}
\label{sec:basevehicle}

The hardware base is a dolly manufactured by Parator Industri AB \cite{paratorAB}. The steering system is based around the \gls{ETS} developed and built by \gls{VSE} \cite{vse} with two hydraulically steerable axles. It was originally meant to be used in trailer steering as an after-market system and thus does not tie in with any of the truck's communication networks or sensor data. This makes it manufacturer independent and very robust. The \gls{ETS} solely relies on the articulation angle between the leading and following unit and the speed of the combination. The articulation angle is obtained via a dedicated sensor mounted on the king-pin of the respective unit, the speed-signal is gathered from the ISO-11992 \gls{CAN}. 

%The axles permit a maximum steering-angle of +/-24$^{\circ}$for both axles. 
\subsection{Rapid-Prototyping System}

To execute the previously developed algorithms, that govern the steering of the \gls{LCV}-combination, they needed to be ported to a platform, capable of interacting with the dolly and the tractor, while ensuring robust behavior during run-time. It was decided to incorporate the \gls{MABII} \cite{mabii} by dSPACE \cite{dspace}, a real-time platform for its advantages in automotive environments with a vast selection of in- and outputs for interfacing with vehicular communications systems (\gls{CAN}, Ethernet, FlexLink). It conveniently ties in with Simulink, which was used for algorithm development, for code-generation. %Furthermore it physically is very robust and has good logging possibilities.
 The tool-chain furthermore comes with the supporting tool ControlDesk to easily provide logging and monitoring during run-time as well as control over the simulation variables' states.


\subsection{Vehicle Dynamics Simulation}
To evaluate the dynamic performance of the \gls{LCV} on vehicle level Volvo Group Truck Technolgy's \gls{VTM} library came to use. It is a library developed in and for Simulink environment and permits the simulation of truck dynamics based on a multi-body model for the kinematic relation and a parametrized tire model using the magic formula.

%\subsection{Arduino}

%To convert from \gls{CAN}-messages to serial data directly usable in the real-time simulation, an Arduino Due\cite{due} together with a MCP2551 \gls{CAN}-transceiver was used. This provided the feedback-loop from the hardware-system to the simulation environment during the \gls{HIL}-testing.


\end{document}